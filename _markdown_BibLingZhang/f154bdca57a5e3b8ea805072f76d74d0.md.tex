\markdownRendererUlBeginTight
\markdownRendererUlItem A First page of summary of variables\markdownRendererUlItemEnd 
\markdownRendererUlEndTight \markdownRendererInterblockSeparator
{}\markdownRendererHorizontalRule{}\markdownRendererInterblockSeparator
{}\markdownRendererHeadingThree{Time-varying System}\markdownRendererInterblockSeparator
{}\markdownRendererHeadingFour{Time-varying arrival rates}\markdownRendererInterblockSeparator
{}There ought to be some more on specific systems\markdownRendererInterblockSeparator
{}\markdownRendererUlBeginTight
\markdownRendererUlItem 26, provides background for the fact that how time-varying arrival rates, which commonly occurs in application but makes performance analysis difficult.\markdownRendererUlItemEnd 
\markdownRendererUlItem 17\markdownRendererUlItemEnd 
\markdownRendererUlItem 50,53\markdownRendererUlItemEnd 
\markdownRendererUlItem 14, 15, service systems typically have arrival rates that vary significantly over time, and the results dramatically reveal the consequence, e.g., showing how the peak congestion lags behind the peak arrival rate, as discussed for the \markdownRendererDollarSign{}M\markdownRendererUnderscore{}t/GI/\markdownRendererBackslash{}infty\markdownRendererDollarSign{} stochastic model.\markdownRendererUlItemEnd 
\markdownRendererUlEndTight \markdownRendererInterblockSeparator
{}\markdownRendererHeadingFour{Time-varying staffing (decisions?)}\markdownRendererInterblockSeparator
{}This is about the decisions. Staffing is of course one of the most obvious decision in call-centers. What are the other decisions? production rate, pricing, etc.\markdownRendererInterblockSeparator
{}\markdownRendererHeadingFour{Time-varying performance functions}\markdownRendererInterblockSeparator
{}\markdownRendererHorizontalRule{}\markdownRendererInterblockSeparator
{}\markdownRendererHeadingThree{Non-exponential informations}\markdownRendererInterblockSeparator
{}\markdownRendererUlBeginTight
\markdownRendererUlItem 79, queueing model in patient flow management\markdownRendererUlItemEnd 
\markdownRendererUlItem 20, how impatient customer will significantly alter system performance. This translates to how production expiration/ demand abandonment will effect system decisions. It proposed MSHT ED and QD regimes.\markdownRendererUlItemEnd 
\markdownRendererUlItem 20, 81, customer abandonment is now recognized as an important feature in service systems. Review that production expiration and demand abandonment is an important feature. Splitting is an important feature we considered in the model.\markdownRendererUlItemEnd 
\markdownRendererUlItem 7, service system often do have non-exponential service and patience distributions. (\markdownRendererEmphasis{need to specify in each cases?} \markdownRendererStrongEmphasis{Due to the complexity in each practical problem, we need to balance problem praticality and model generality})\markdownRendererUlItemEnd 
\markdownRendererUlItem 46,76,77,81 shows that the patience distribution beyond its mean has a significant impact. However, 76, 77 also showns that the steady-state performance in the stationary \markdownRendererDollarSign{}G/GI/s+GI\markdownRendererDollarSign{} model is relatively insensitive to the service-time cdf beyond its mean.\markdownRendererUlItemEnd 
\markdownRendererUlItem In Dr. Liu's 2012 paper, they have shwon the importance of information beyond its mean can have a significant impact as well as for the transient performance.\markdownRendererUlItemEnd 
\markdownRendererUlItem 31, an example of application of the result in Dr. Liu's 2012 paper(chapter 2), create new effective real-time delay predictors for arriving customers in a service system with time-varying arrivals.\markdownRendererUlItemEnd 
\markdownRendererUlEndTight \markdownRendererInterblockSeparator
{}\markdownRendererHeadingFour{Classical Erlang Models}\markdownRendererInterblockSeparator
{}Since we have a lot erlang assumptions in two of my models. It is important to review that and \markdownRendererStrongEmphasis{clarify that due to the complexity of our model and the complexity of the associated optimization problem we used Erlang model.}\markdownRendererInterblockSeparator
{}\markdownRendererHeadingFour{Non-exponential}\markdownRendererInterblockSeparator
{}Specifically in the car-sharing project, I need to justify the importance of including \markdownRendererEmphasis{non-exponential} and \markdownRendererEmphasis{distributional} information.\markdownRendererInterblockSeparator
{}\markdownRendererUlBeginTight
\markdownRendererUlItem Cathy Xia Paper, the differences is that we considered a new problem and included distributional information\markdownRendererUlItemEnd 
\markdownRendererUlEndTight \markdownRendererInterblockSeparator
{}\markdownRendererHorizontalRule{}\markdownRendererInterblockSeparator
{}\markdownRendererHeadingThree{Fluid Model}\markdownRendererInterblockSeparator
{}To discuss the fluid approximation, one needs to discuss both the single-server and many-server model. The former one is referred to as the \markdownRendererEmphasis{conventional} heavy traffic regime. The many-server model, from a math point of view, has \markdownRendererEmphasis{infinitely} many servers in the pre-limit as system \markdownRendererEmphasis{scale} increases to infinity.\markdownRendererInterblockSeparator
{}\markdownRendererUlBeginTight
\markdownRendererUlItem In his third chapter, he mentioned the following. \markdownRendererEmphasis{Fluid model is tractable, we are providing the basis for creating a performance-analysis tool for large-scale service systems, like the Queueing Network Analyzer(QNA) in 73 and 8. Algorithms based on performance formulas are appealing to supplement and complement computer simulation, because the models can be created and solved much more quickly. Thus they can be applied quickly in what if studies. They also can be efficiently embedded in optimization algorithms to systematical determine design and control parameters to meet performance objectives.} Therefore, my work follows closely to the development of time-varying fluid model and furthermore to incorporate it in a optimization problem/ optimal control problem.\markdownRendererUlItemEnd 
\markdownRendererUlEndTight \markdownRendererInterblockSeparator
{}\markdownRendererHeadingFour{Application of Deterministic Fluid Models}\markdownRendererInterblockSeparator
{}\markdownRendererUlBeginTight
\markdownRendererUlItem 57, insight of fluid model as a legitimate model itself\markdownRendererUlItemEnd 
\markdownRendererUlItem 54, 29, shown a long history of applying deterministic fluid models\markdownRendererUlItemEnd 
\markdownRendererUlItem \markdownRendererUnderscore{}\markdownRendererEmphasis{In this thesis, we primarily do not concern with establishing limits for sequences of scaled queueing processes. We directly concerned with the fluid model itself. \markdownRendererUnderscore{}}\markdownRendererUlItemEnd 
\markdownRendererUlItem We do not concern establishing limits. We construct stochastic optimization problem and use fluid model as an approximation. More importantly, we show the stochastic optimization problem equipped with fluid optimal control is a lower bound and asymptotically optimal\markdownRendererUlItemEnd 
\markdownRendererUlEndTight \markdownRendererInterblockSeparator
{}\markdownRendererHeadingFour{Heavy-traffic single-server}\markdownRendererInterblockSeparator
{}\markdownRendererUlBeginTight
\markdownRendererUlItem 74 Heavy-traffic fluid and diffusion approximation become helpful. Due to the fact that we used single-server fluid model. We may need to dig a little deeper. Specifically, \markdownRendererDollarSign{}\markdownRendererBackslash{}cdot/G\markdownRendererEmphasis{t/\markdownRendererBackslash{}infty\markdownRendererDollarSign{}, \markdownRendererDollarSign{}\markdownRendererBackslash{}cdot/M}t/1\markdownRendererDollarSign{}, double-ended queue model, \markdownRendererDollarSign{}M\markdownRendererUnderscore{}t/M/1\markdownRendererDollarSign{}.\markdownRendererUlItemEnd 
\markdownRendererUlItem 74 has been mentioned at the very start as an extensive account\markdownRendererUlItemEnd 
\markdownRendererUlItem 38, introduced the \markdownRendererDollarSign{}GI/GI/1\markdownRendererDollarSign{} model.\markdownRendererUlItemEnd 
\markdownRendererUlItem \markdownRendererEmphasis{\markdownRendererEmphasis{Who developed the \markdownRendererDollarSign{}M}t/M/1\markdownRendererDollarSign{} mode?}\markdownRendererUnderscore{} Dr. Liu in his 2012 paper. \markdownRendererStrongEmphasis{Need to check that!}\markdownRendererUlItemEnd 
\markdownRendererUlItem 5,32 extended the conventional heavy-traffic to queues with multiple servers. Furthermore, in 32 the convergence of the entire queue-length process is established.\markdownRendererUlItemEnd 
\markdownRendererUlEndTight \markdownRendererInterblockSeparator
{}\markdownRendererHeadingFour{Heavy-traffic many-server}\markdownRendererInterblockSeparator
{}\markdownRendererEmphasis{Do I need to discuss the differences between my policy and the QED regmine?} Although these two different policies are proposed under different framework.\markdownRendererInterblockSeparator
{}\markdownRendererUlBeginTight
\markdownRendererUlItem 28, Halfin and Whitt regime\markdownRendererUlItemEnd 
\markdownRendererUlItem 20, extened Halfin and Whitt regime to Erlang A model\markdownRendererUlItemEnd 
\markdownRendererUlItem 77\markdownRendererUlItemEnd 
\markdownRendererUlItem 20,46,55,56, Fluid model arises in MSHT\markdownRendererUlItemEnd 
\markdownRendererUlEndTight \markdownRendererInterblockSeparator
{}\markdownRendererHeadingFour{Heavy-traffic many-server with time-varying rate}\markdownRendererInterblockSeparator
{}\markdownRendererUlBeginTight
\markdownRendererUlItem 46, 47, 48, a theoretical basis for the MSHT limits for models with time-varying arrival rates and staffing\markdownRendererUlItemEnd 
\markdownRendererUlItem Dr. Liu's 2012 paper and the one he has on \markdownRendererDollarSign{}G\markdownRendererEmphasis{t/GI/s}t+GI\markdownRendererDollarSign{}\markdownRendererUlItemEnd 
\markdownRendererUlEndTight \markdownRendererInterblockSeparator
{}\markdownRendererHeadingThree{Queueing Network (as is mentioned by Dr. Cao)}\markdownRendererInterblockSeparator
{}Claim that single queue is apparently not capturing all cases. There are problems can be modeled with different structures. Need to review at least, \markdownRendererEmphasis{double-ended queue}, \markdownRendererEmphasis{queueing network}, and \markdownRendererEmphasis{tandem queue}.\markdownRendererInterblockSeparator
{}\markdownRendererUlBeginTight
\markdownRendererUlItem 10, Jackson Network (That is all that he reviewed. Maybe we could review a bit more, the \markdownRendererEmphasis{V}, \markdownRendererEmphasis{reverse V} and the \markdownRendererEmphasis{N} model and pay close attention if any control problem is done.)\markdownRendererUlItemEnd 
\markdownRendererUlItem \markdownRendererUlItemEnd 
\markdownRendererUlEndTight \markdownRendererInterblockSeparator
{}\markdownRendererHeadingFour{Transient and Asymptotic Performance}\markdownRendererInterblockSeparator
{}Isn't transient in our definition the same as time-varying. And I think we should separate the review on transient and asymptotic performance.\markdownRendererInterblockSeparator
{}\markdownRendererUlBeginTight
\markdownRendererUlItem 46-48, 56, 58, analyzed transient dynamics, fluid and diffusion approximation.\markdownRendererUlItemEnd 
\markdownRendererUlItem Specifically, 58, \markdownRendererDollarSign{}M\markdownRendererEmphasis{t/M}t/k\markdownRendererEmphasis{t+M}t\markdownRendererDollarSign{} queue is analyzed\markdownRendererUlItemEnd 
\markdownRendererUlItem 56, MSHT limits or infinite-server queue\markdownRendererUlItemEnd 
\markdownRendererUlItem 3, for a system with constant parameters, we care about the steady-state models ad they provide evaluation of the average system costs and revenue.\markdownRendererUlItemEnd 
\markdownRendererUlEndTight \markdownRendererInterblockSeparator
{}\markdownRendererHorizontalRule{}\markdownRendererInterblockSeparator
{}\markdownRendererHeadingFour{Introduction to Chapter 2}\markdownRendererInterblockSeparator
{}\markdownRendererUlBeginTight
\markdownRendererUlItem 56, MSHT limit for the infinite server model\markdownRendererUlItemEnd 
\markdownRendererUlItem 45, MSHT limit for the model with general service distribution\markdownRendererUlItemEnd 
\markdownRendererUlItem 36, 37, 56, 62, new limits in Chapter 5 are consistent with recent\markdownRendererUlItemEnd 
\markdownRendererUlEndTight \markdownRendererInterblockSeparator
{}\markdownRendererHorizontalRule{}\markdownRendererInterblockSeparator
{}\markdownRendererHeadingThree{A network Generalization}\markdownRendererInterblockSeparator
{}\markdownRendererHeadingFour{Dr. Liu's network review}\markdownRendererInterblockSeparator
{}\markdownRendererUlBeginTight
\markdownRendererUlItem Need to check with Dr. Liu what does his chapter three turns into. The difference between his model and mine are the following: \markdownRendererEmphasis{closed network} and \markdownRendererEmphasis{routing}. The first-come first-server can be regard as a proportional routing, however we have shown that it is under some assumptions not always the best policy to implement.\markdownRendererUlItemEnd 
\markdownRendererUlItem 50,\markdownRendererUlItemEnd 
\markdownRendererUlEndTight \markdownRendererInterblockSeparator
{}\markdownRendererHeadingFour{Mine}\markdownRendererInterblockSeparator
{}I should include a slightly more comprehensive review, including the \markdownRendererEmphasis{V model}, \markdownRendererEmphasis{reverse V model}, and \markdownRendererEmphasis{N model}, \markdownRendererEmphasis{tandem model}.\relax